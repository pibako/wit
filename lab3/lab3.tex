
%%% Local Variables: 
%%% mode: latex
%%% TeX-master: t
%%% End: 
\documentclass[12pt,letterpaper]{article}

\usepackage[polish]{babel}
\usepackage[utf8]{inputenc}
\usepackage{polski}
\usepackage[T1]{fontenc}
\frenchspacing
\usepackage{indentfirst}

\usepackage{amsmath} % just math
% \usepackage{amssymb} % allow blackboard bold (aka N,R,Q sets)

\usepackage{hyperref} %hyperlinks

\usepackage{ulem}
\linespread{1.6}  % double spaces lines
\usepackage[left=1in,top=1in,right=1in,bottom=1in,nohead]{geometry}
\begin{document}

\linespread{1} % single spaces lines
\small \normalsize %% dumb, but have to do this for the prev to work
\begin{flushright}
  Laboratorium programowania w języku Java -- 3 
  \footnote{Na podstawie kursu Uniwersytetu Warszawskiego: \url{http://wazniak.mimuw.edu.pl} }\\
  Piotr Kowalski,
  \today
\end{flushright}

\section{Wyszukiwanie i plik - dokończenie z poprzednich zajęć.}
Teraz gdy mamy implementację klas reprezentujących kształty oraz zbior kształtów należy zaimplementować dwie metody w klasie \verb+Shapes+ (nie pomylić z klasą \verb+Shape+). 

\begin{enumerate}
\item \verb+findLargest+ - należy znaleźć figury o największym polu i utworzyć listę a nastepnie przekazać ją jako wartość zwracaną.
\item \verb+readShapesFromFile+ - Aby ułatwić sobie tworzenie obiektów
  do projektu dołączony jest plik z deklaracją kształtów. Należy go
  wczytać i utworzyć obiekt ShapeSet zawierający kształty występujące
  w pliku. Należy wczytać wszystkie obiekty linia po linii przy pomocy
  np. \verb+BufferedReader+ a następnie sparsować je wykorzystując
  metodę klasy \verb+String split+.
  \begin{enumerate}
  \item Pomyśl jak zabezpieczyć działanie programu aby uniknąć
    wczytywania źle sformatowanych danych.

  \item Java Tutorials I/O - \url{http://docs.oracle.com/javase/tutorial/essential/io/streams.html}
  \end{enumerate}

\item Pamiętaj o testowaniu każdej metody.
\end{enumerate}

\section{Kilka prostych zadań}
\begin{enumerate}
\item Napisz program, który czyta plik tekstowy \verb+fun.txt+ znak po
  znaku i wypisuje jego zawartość na konsolę. Do wypisywania użyj
  obiektu klasy \verb+PrintWriter+ lub \\ \verb+BufferedWriter+
  (zobacz \verb+OutputStreamWriter+).
\item Zmodyfikuj rozwiązanie poprzedniego zadania, tak aby program
  pytał o nazwę pliku do wypisania oraz czy wypisywać kolejny
  plik. Obsłuż mogące się pojawiać wyjątki.
\item Przepisz plik \verb+fun.txt+ do postaci binarnej z użyciem \verb+DataOutputStream+
\end{enumerate}

\section{Wczytaj i narysuj}
Zgadnij jaka funkcja znajduje się w pliku \verb+fun.txt+.
1-wsza kolumna wartość na osi odciętych (X), 2-ga kolumna wartość funkcji na
osi rzędnych (Y). Zmodyfikuj lub dodaj nową funkcję na bazie
\verb+XYSeries createDataSeries()+ znajdującej się w \verb+DataVisualizer.java+.

\paragraph{}

Do tego zadania wykorzystana zostanie biblioteka JFreeChart. Aby z
niej korzystać należy:
\begin{enumerate}
\item pobrać ze strony:
  \url{http://www.jfree.org/jfreechart/download.html} aktualną wersję
  biblioteki,
\item przekopiować katalog \verb+lib+ wraz z zawartością do np:
  \verb+/home/user/NetBeansProjects/+,
\item w NetBeans wejść do \verb+->Tools->Libraries+ i dodać nową
  bibliotekę \verb+JFreeChart+,
\item w zakładce Classpath dla tej biblioteki należy podać ścieżkę do
  jfreechart-1.0.13.jar oraz jcommon-1.0.16.jar znajdujących się w
  \verb+lib+,
\item w zakładce Javadoc należy dodać url:
  \url{http://www.jfree.org/jfreechart/api/javadoc/},
\item we właściwościach projektu (\verb+Properties+) należy w
  kategorii \verb+Libraries+ dodać naszą nową bibliotekę w zakładce \verb+Compile+. 
\end{enumerate}

\end{document}
