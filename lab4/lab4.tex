
%%% Local Variables: 
%%% mode: latex
%%% TeX-master: t
%%% End: 
\documentclass[12pt,letterpaper]{article}

\usepackage[polish]{babel}
\usepackage[utf8]{inputenc}
\usepackage{polski}
\usepackage[T1]{fontenc}
\frenchspacing
\usepackage{indentfirst}

\usepackage[usenames,dvipsnames]{color} % just color
\usepackage{listings} % code listing
\lstset{
  language=java,
  basicstyle=\footnotesize,
  numbers=left,
  breaklines=true,
  frame=leftline,
  keywordstyle=\color[rgb]{0,0,1},
  commentstyle=\color[rgb]{0.133,0.545,0.133},
  stringstyle=\color[rgb]{0.627,0.126,0.941}
}

\usepackage{amsmath} % just math
% \usepackage{amssymb} % allow blackboard bold (aka N,R,Q sets)

\usepackage{hyperref} %hyperlinks

\usepackage{ulem}
\linespread{1.6}  % double spaces lines
\usepackage[left=1in,top=1in,right=1in,bottom=1in,nohead]{geometry}
\begin{document}

\linespread{1} % single spaces lines
\small \normalsize %% dumb, but have to do this for the prev to work
\begin{flushright}
  Laboratorium programowania w języku Java -- 4 
  \footnote{Na podstawie kursu Uniwersytetu Warszawskiego: \url{http://wazniak.mimuw.edu.pl} }\\
  Piotr Kowalski,
  \today
\end{flushright}

\section{Wczytaj i narysuj}
Zgadnij jaka funkcja znajduje się w pliku \verb+fun.txt+.
1-wsza kolumna wartość na osi odciętych (X), 2-ga kolumna wartość funkcji na
osi rzędnych (Y). Zmodyfikuj lub dodaj nową funkcję na bazie
\verb+XYSeries createDataSeries()+ znajdującej się w .

\paragraph{}

Do tego zadania wykorzystana zostanie biblioteka JFreeChart. Aby z
niej korzystać należy:
\begin{enumerate}
\item pobrać ze strony:
  \url{http://www.jfree.org/jfreechart/download.html} aktualną wersję
  biblioteki,
\item przekopiować katalog \verb+lib+ wraz z zawartością do np:
  \verb+/home/user/NetBeansProjects/+,
\item w NetBeans wejść do \verb+->Tools->Libraries+ i dodać nową
  bibliotekę \verb+JFreeChart+,
\item w zakładce Classpath dla tej biblioteki należy podać ścieżkę do
  jfreechart-1.0.13.jar oraz jcommon-1.0.16.jar znajdujących się w
  \verb+lib+,
\item w zakładce Javadoc należy dodać url:
  \url{http://www.jfree.org/jfreechart/api/javadoc/},
\item we właściwościach projektu (\verb+Properties+) należy w
  kategorii \verb+Libraries+ dodać naszą nową bibliotekę w zakładce \verb+Compile+. 
\end{enumerate}

\paragraph{}
Przeanalizuj poniższy kod (\verb+DataVisualizer+) w kontekście tego
zadania:
\begin{lstlisting}[language=Java]
// Swing - GUI
JFrame frame = new JFrame("JFreeChart chart!");
ChartPanel chartPanel = new ChartPanel(chart);
frame.getContentPane().add(chartPanel);
frame.setSize(new Dimension(600, 400));
frame.setLocation(300, 200);
frame.setDefaultCloseOperation(JFrame.EXIT_ON_CLOSE);
frame.setVisible(true);
\end{lstlisting}

\section{Tworzenie okna poprzez agregacje (kompozycje)}
\label{sec:tworz-okna-agreg}

Utwórz nowy projekt o nazwie Windows (nie to nie ma nic wspólnego z \verb+M$ Windows+;).
Utwórz statyczną, publiczną metodę \verb+createGUI()+, oraz w metodzie \verb+main()+
wywołaj ją poprzez:
\begin{lstlisting}[language=Java]
/* Magiczna sztuczka... GUI popwinno byc zawsze tworzone w ten sposob.
 * invokeLater przekazuje nowo utworzony watek klasy anonimowej do 
 * watku obslugi zdarzen. 
 * Zasada jest prosta - wszystko co dotyczy GUI powinno byc
 * obslugiwane w watku obslugi zdarzen.
 */
javax.swing.SwingUtilities.invokeLater(new Runnable() {
  public void run() {
    createGUI();
  }
});
\end{lstlisting}

Wzorując się na poprzednim zadaniu zobacz co się stanie,
gdy nie zostanie ustawiona domyślna akcja zamknięcia
okna. Wykorzystując klasę \verb+Toolkit+ oraz jej metodę statyczną \verb+getDefaultToolkit()+
 narysuj okno o rozmiarach:
 \begin{enumerate}
 \item szerokości = 0.5 * szerokość twojego ekranu (w pikselach),
 \item wysokości = 0.5 * wysokość twojego ekranu,
 \item oraz początkowej pozycji lewego górnego rogu (x = 0.25 szerokości ekranu, y
   = 0.25 wysokości ekranu),
 \end{enumerate}
czyli idealnie na środku ekranu.

\section{ Tworzenie okna poprzez dziedziczenie z klasy JFrame}
\label{sec:tworz-okna-dziedzi}

Utwórz nową klasę w projekcie \verb+Windows+ o nazwie
\verb+SubclassWindow+. Poprzednie zadanie zmodyfikuj tak aby klasa
\verb+JFrame+ rozszerzała (\verb+extends+) klasę
\verb+SubclassWindow+. Uruchom aplikację dodając w klasie \verb+SubclassWindow+
metodę \verb+main+ oraz tworząc nową konfigurację w  NetBeans IDE.

\section{Układy (Layout)}
Domyślnym układem jest \verb+BorderLayout+, zapoznaj się z jego
dokumentacją:
\url{http://download.oracle.com/javase/6/docs/api/java/awt/BorderLayout.html}
\begin{enumerate}
\item Wybierz jedno z poprzednich zadań i dodaj do okna dwa przyciski:
  jeden na górze (''UP'') i jeden na dole ("DOWN"). Po środku wstaw
  etykietę ("0").
\item Zmień layout na FlowLayout (zobacz:
  \url{http://download.oracle.com/javase/6/docs/api/java/awt/FlowLayout.html}). 
Co się zmieniło?
\item Layout'y można łączyć\ldots popróbuj z różnymi układami:
  \verb+GridLayout+, \verb+BoxLayout+, \verb+CardLayout+\ldots
  (zobacz: \url{http://download.oracle.com/javase/tutorial/uiswing/layout/layoutlist.html})
\end{enumerate}
Zwykle GUI tworzone jest jednak z wykorzystaniem środowisk graficznych
typu NetBeans IDE. Co znacznie ułatwia ale nie zwalnia całkiem ze
znajomości \verb+Layout+'ów.

\section{Obsługa zdarzeń}
\label{sec:obsuga-zdarze}

Dodaj do poprzedniego programu obsługę zdarzeń myszy. Tak aby po
wciśnięciu ''up'' wartość na wyświetlaczu JLabel po środku wzrosła o 1
a po wciśnięciu ''down'' zmalała o 1.
\begin{enumerate}
\item Obsługa rzycisku \verb+up+ powinna być wykonana przy pomocy
  implementacji klasy anonimowej \verb+MouseAdapter+
\item Obsługa rzycisku \verb+down+ powinna być wykonana przy pomocy
  klasy wewnętrznej \verb+DownButtonEvents+ implementującej
  \verb+MouseListener+
\item Co by się stało gdybyś dodał bądź do klasy \verb+SubclassWindow+
  lub \verb+Windows+ rozszerzenie o metody interfejsu \verb+implements MouseListener+? 
\end{enumerate}


\section{Menu}
\label{sec:menu}

Dodaj menu do głównego okna utworzonego programu (zobacz:
\url{http://download.oracle.com/javase/tutorial/uiswing/components/menu.html}).
Wykorzystaj: \verb+JMenuBar+, \verb+JMenu+, \verb+JMenuItem+.
\begin{enumerate}
\item Dodaj dwa \verb+JMenuItem+'y ''up'' i ''down''
\item Podłącz tą samą obsługę zdarzeń co przyciski ''up'' i ''down''.
\end{enumerate}


\end{document}
