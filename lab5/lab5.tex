
%%% Local Variables: 
%%% mode: latex
%%% TeX-master: t
%%% End: 
\documentclass[12pt,letterpaper]{article}

\usepackage[polish]{babel}
\usepackage[utf8]{inputenc}
\usepackage{polski}
\usepackage[T1]{fontenc}
\frenchspacing
\usepackage{indentfirst}

\usepackage[usenames,dvipsnames]{color} % just color
\usepackage{listings} % code listing
\lstset{
  language=java,
  basicstyle=\footnotesize,
  numbers=left,
  breaklines=true,
  frame=leftline,
  keywordstyle=\color[rgb]{0,0,1},
  commentstyle=\color[rgb]{0.133,0.545,0.133},
  stringstyle=\color[rgb]{0.627,0.126,0.941}
}

\usepackage{amsmath} % just math
% \usepackage{amssymb} % allow blackboard bold (aka N,R,Q sets)

\usepackage{hyperref} %hyperlinks

\usepackage{ulem}
\linespread{1.6}  % double spaces lines
\usepackage[left=1in,top=1in,right=1in,bottom=1in,nohead]{geometry}
\begin{document}

\linespread{1} % single spaces lines
\small \normalsize %% dumb, but have to do this for the prev to work
\begin{flushright}
  Laboratorium programowania w języku Java -- 5 
  % \footnote{Na podstawie kursu Uniwersytetu Warszawskiego:
  % \url{http://wazniak.mimuw.edu.pl} }
  \\
  Piotr Kowalski,
  \today
\end{flushright}

\section{Model-View-Controller}
\label{sec:mvc}

Model-View-Controller chyba po raz pierwszy opisany przez twórcę {\em
  Trygve Reenskaug}'a na przykładzie języka \verb+Smalltalk+. Jest jednym z
najważniejszych wzorców projektowych i koncepcji wykorzystywanych przy
tworzeniu aplikacji z GUI. Służy rozdzieleniu warstwy widoku aplikacji
od jej mózgu (modelu). Zaimplementowany i wykorzystywany m.in. w:
\begin{itemize}
\item Cocoa (Mac OS X Framework) -- \verb+Objective-C+
\item JSF, Spring, Struts, Oracle Application Framework, etc. -- \verb+Java+
\item Dojo -- \verb+JavaScript+
\item ASP.NET MVC Framework -- \verb+ASP+
\item Joomla, Zend, Symfony, DooPHP -- \verb+PHP+
\item Django -- \verb+Python+
\item Ruby on Rails -- \verb+Rails+
\end{itemize}

\section{Temperatura}
\label{sec:temperatura}

Zadanie na dziś:
\begin{itemize}
\item zapoznać się z treścią artykułu:
  \url{http://csis.pace.edu/~bergin/mvc/mvcgui.html}
\item napisać program który będzie te same dane prezentował oraz pozwalał je modyfikować w różny sposób,
\item dodać nową wizualizację (zamiast słupka rtęci narysować wykres kołowy),
\item zmienić implementację aby można było zmieniać miasto dla którego
  wyświetlamy temperaturę, w tym celu należy:
  \begin{enumerate}
  \item zmodyfikować model aby przechowywał informację o miejscach i
    odpowiadającej jej temperaturze
  \item dodać nowy widok z combo-box'em i możliwością zmiany miejscowości.
  \end{enumerate}

\end{itemize}
\end{document}
