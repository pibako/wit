
%%% Local Variables: 
%%% mode: latex
%%% TeX-master: t
%%% End: 
\documentclass[12pt,letterpaper]{article}

\usepackage[polish]{babel}
\usepackage[utf8]{inputenc}
\usepackage{polski}
\usepackage[T1]{fontenc}
\frenchspacing
\usepackage{indentfirst}

\usepackage[usenames,dvipsnames]{color} % just color
\usepackage{listings} % code listing
\lstset{
  language=java,
  basicstyle=\footnotesize,
  numbers=left,
  breaklines=true,
  frame=leftline,
  keywordstyle=\color[rgb]{0,0,1},
  commentstyle=\color[rgb]{0.133,0.545,0.133},
  stringstyle=\color[rgb]{0.627,0.126,0.941}
}

\usepackage{amsmath} % just math
% \usepackage{amssymb} % allow blackboard bold (aka N,R,Q sets)

\usepackage{hyperref} %hyperlinks

\usepackage{ulem}
\linespread{1.6}  % double spaces lines
\usepackage[left=1in,top=1in,right=1in,bottom=1in,nohead]{geometry}
\begin{document}

\linespread{1} % single spaces lines
\small \normalsize %% dumb, but have to do this for the prev to work
\begin{flushright}
  Laboratorium programowania w języku Java -- 6
  \footnote{Na podstawie materiałów Oracle 
    \url{http://download.oracle.com/javase/tutorial/networking/TOC.html} }
  \\
  Piotr Kowalski,
  \today
\end{flushright}

\section{Programowanie aplikacji sieciowych.}
\label{sec:programowanie-aplikacji-sieciowych}


\begin{enumerate}
\item Większość potrzebnych informacji do zroumienia programowania
  aplikacji sieciowych można znaleźć pod adresem:
  \url{http://download.oracle.com/javase/tutorial/networking/TOC.html}
\item aby szybko pochwycić sposób tworzenia wątków warto zajrzeć tutaj:
  \url{http://download.oracle.com/javase/tutorial/essential/concurrency/runthread.html}.
\end{enumerate}
Do dzisiejszych zajęć dołączone zostały trzy projekty. \verb+MVC+ z
poprzednich zajęć, \verb+temperature_server+ oraz \verb+temperature_client+.

\section{Klient-Serwer}
\label{sec:klient-serwer}

Pierwsze zadanie polega na uruchomieniu dwóch programów
(\verb+temperature_server+ oraz \verb+temperature_client+) i analizie ich
działania. Jak działają, jak się komunikują? Czy mogą istnieć
niezależnie? Czy istotna jest kolejnośc uruchomienia?

\section{Czujnik temperatury}
\label{sec:czujnik-temperatury}

Zadanie polega na wykorzystaniu \verb+MVC+ z poprzednich zajęć i
dopisaniu nowego widoku (kilent serwera temperatury z poprzedniego zadania), który będzie aktualizował model.



\end{document}
